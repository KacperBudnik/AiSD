\documentclass[50pt]{mwart}
\usepackage{polski}
\usepackage[utf8]{inputenc}
\usepackage[T1]{fontenc}
\usepackage{lmodern}
\usepackage{mathtools,amsthm,amssymb,icomma,upgreek,xfrac,enumitem,multicol,paracol,esvect}
%\usepackage{bigints}
%\usepackage{relsize,exscale}
%\usepackage{tabularht}
\usepackage[margin=2cm]{geometry}
%%\geometry{a4paper, left=0mm, right=0mm, top=0mm}
%\usepackage{paracol}
%\usepackage{array}



\begin{document}

\section{Grupa 1}
\begin{align*}
f_1(n) &= n^{0.999999}\log n\\ 
 f_2(n) &= 10000000n \\
 f_3(n) &= 1.000001^n\\
 f_4(n) &= n^2 
\end{align*}
		

Twierdzę, że
$$ f_1\preccurlyeq f_2\preccurlyeq f_4 \preccurlyeq f_3$$

\begin{proof}

$f_1\preccurlyeq f_2$

\noindent$ \lim\limits_{n\to\infty} \frac{f_1(n)}{f_2(n)} = \lim\limits_{n\to\infty} \frac{n^{0.999999}\log n}{10000000n} = \lim\limits_{n\to\infty} \frac{ n*n^{-10^6}\log n}{10^7n} = \lim\limits_{n\to\infty} \frac{1}{10^7} \frac{\ln n}{n^{10^{-6}}} \overset{H}{=} \lim\limits_{n\to\infty} \frac{1}{10} \frac{\frac{1}{n}}{n^{10^{-6}-1}}= \frac{1}{10} \lim\limits_{n\to\infty} \frac{1}{n^{10^{-6}}}=0 $\newline

$f_2\preccurlyeq f_4$

\noindent$ \lim\limits_{n\to\infty} \frac{f_2(n)}{f_4(n)} =  \lim\limits_{n\to\infty} \frac{10^7n}{n^2} \overset{H}{=} \lim\limits_{n\to\infty} \frac{10^7}{2n} = 0$\newline


$f_4\preccurlyeq f_3$

\noindent$ \lim\limits_{n\to\infty} \frac{f_4(n)}{f_3(n)} = \lim\limits_{n\to\infty} \frac{n^2}{1.000001^n} \overset{H}{=} \lim\limits_{n\to\infty} \frac{2n}{1.000001^n\ln(1.000001)} \overset{H}{=} \lim\limits_{n\to\infty} \frac{2}{1.000001^n\ln^2(1.000001)} = 0$\newline
\end{proof}

\section{Grupa 2}
\begin{align*}
f_1(n) &= 2^{100n}\\
f_2(n) &= {n\choose2}\\
 f_3(n) &= n\sqrt n
\end{align*}

Twierdzę, że
$$ f_3\preccurlyeq f_2 \preccurlyeq f_1$$

\begin{proof}
	$ f_3\preccurlyeq f_1 $
	
	\noindent$ \lim\limits_{n\to\infty} \frac{f_3(n)}{f_2(n)} = \lim\limits_{n\to\infty} \frac{n\sqrt n}{{n\choose2}} = \lim\limits_{n\to\infty} \frac{n\sqrt n}{\frac{n!}{2!(n-2)!}} = \lim\limits_{n\to\infty} \frac{n\sqrt n}{n(n-1)} \overset{H}{=} \lim\limits_{n\to\infty} \frac{3\sqrt n}{2n - 1} \overset{H}{=} \lim\limits_{n\to\infty} \frac{3}{4\sqrt n} = 0$\newline
	
	$ f_2 \preccurlyeq f_1 $
	
	\noindent$ \lim\limits_{n\to\infty} \frac{f_2(n)}{f_1(n)} = \lim\limits_{n\to\infty} \frac{{n\choose2}}{2^{100n}} = \lim\limits_{n\to\infty} \frac{\frac{n!}{2!(n-2)!}}{2^{100n}} = \lim\limits_{n\to\infty} \frac{n(n-1)}{2^{100n+1}} \overset{H}{=} \lim\limits_{n\to\infty} \frac{2n - 1}{2^{100n+1}\ln2\cdot100} \overset{H}{=} \lim\limits_{n\to\infty} \frac{2}{2^{100n+1}\ln^2 2\cdot10^4} = 0$\newline
\end{proof}

\section{Grupa 3}
\begin{align*}
f_1(n) &= n^{\sqrt n}\\
f_2(n) &= 2^n\\
 f_3(n) &= n^{10}2^{n/2} \\
 f_4(n) &= \sum\limits^n_{i=1}(i+1)
\end{align*}

Twierdzę, że
$$ f_4\preccurlyeq f_1\preccurlyeq f_3\preccurlyeq f_2$$

\begin{proof}
	$ f_4\preccurlyeq f_1 $

\noindent$ \lim\limits_{n\to\infty} \frac{f_4(n)}{f_1(n)} = \lim\limits_{n\to\infty} \frac{\sum^n_{i=1}(i+1)}{n^{\sqrt n}} = 0 $, bo
$$ 0 \le \frac{f_4(n)}{f_1(n)} = \frac{\frac{1}{2}n(n+1)}{n^{\sqrt n}} \le \frac{n^2 + n}{2n^3} \le \frac{2n^2}{2n^3}  = \frac{1}{n} \overset{n\to\infty}{\to} 0 $$
i z twierdzenia o 3 ciągach\newline

$ f_1 \preccurlyeq f_3 $

\noindent$ \lim\limits_{n\to\infty} \frac{f_1(n)}{f_3(n)} = \lim\limits_{n\to\infty} \frac{n^{\sqrt n}}{n^{10}2^{n/2}} = \lim\limits_{n\to\infty} \frac{n^{\sqrt n - 10}}{{\sqrt 2}^n} = 0$,bo 
$$0\le \frac{f_1(n)}{f_3(n)} \le \frac{n^{\sqrt n}}{{\sqrt 2}^n} = {\left(\frac{n}{{\sqrt 2}^{\sqrt n}}\right)}^{\sqrt n},$$
a ponieważ
$\lim\limits_{n\to\infty} \frac{n}{{\sqrt 2}^{\sqrt n}} = 0$, to prawa strona dąży do 0 ($0^\infty=0$, symbol oznaczony). Więc z twierdzenia o 3 ciągach szukana granica również dąży do 0.\newline

%\noindent$ \lim\limits_{n\to\infty} \frac{f_1(n)}{f_3(n)} = \lim\limits_{n\to\infty} \frac{n^{\sqrt n}}{n^{10}2^{n/2}} = \lim\limits_{n\to\infty} \frac{\ln(n^{\sqrt n})}{\ln (n^{10}2^{n/2})} = \lim\limits_{n\to\infty} \frac{\sqrt n\ln n}{10\ln n+n\ln(2)/2} \overset{H}{=} \lim\limits_{n\to\infty} \frac{\frac{\ln n}{\sqrt n}}{\frac{10}{n}+\ln(2)/2} \\= \lim\limits_{n\to\infty} \frac{\ln n}{\sqrt n}\cdot \lim\limits_{n\to\infty} \frac{1}{\frac{10}{n}+\ln(2)/2} = \frac{2}{\ln 2}\cdot\lim\limits_{n\to\infty} \frac{\ln n}{\sqrt n} \overset{H}{=} \frac{2}{\ln 2}\cdot\lim\limits_{n\to\infty} \frac{1}{n\frac{1}{2\sqrt{n}}}=\frac{2}{\ln 2}\cdot\lim\limits_{n\to\infty} \frac{2}{\sqrt n}=0$\newline

$ f_3 \preccurlyeq f_2 $

\noindent $ \lim\limits_{n\to\infty} \frac{f_3(n)}{f_2(n)} = \lim\limits_{n\to\infty} \frac{n^{10}2^{n/2}}{2^n} = \lim\limits_{n\to\infty} \frac{n^{10}}{2^{n/2}}$,\newline

\noindent korzystając teraz 10 razy z reguły de l'Hospitala otrzymujemy\newline

\noindent $\lim\limits_{n\to\infty} \frac{f_3(n)}{f_2(n)} \lim\limits_{n\to\infty} \frac{10!}{\ln^{10}(2)2^{n/2}\left(\frac{1}{2}\right)^{10}}=0$

%$ f_1 \preccurlyeq f_2 $
%
%\noindent$ \lim\limits_{n\to\infty} \frac{f_1(n)}{f_2(n)} = \lim\limits_{n\to\infty} \frac{n^{\sqrt n}}{2^n} = \lim\limits_{n\to\infty} \frac{\exp({\ln n \sqrt n})}{\exp({n \ln 2})} = \exp\left(\lim\limits_{n\to\infty} \ln n \sqrt n - n\ln2 \right) = \exp\left(\lim\limits_{n\to\infty} n \left(\frac{\ln n}{\sqrt n\ln 2} - 1\right)\ln 2 \right) \\= 2\exp\left(\lim\limits_{n\to\infty} n \cdot \lim\limits_{n\to\infty} \left(\frac{\ln n}{\sqrt n\ln 2} - 1\right)\right) = 2\exp\left(\lim\limits_{n\to\infty} n \cdot \left(\lim\limits_{n\to\infty}\frac{\ln n}{\sqrt n\ln 2} - \lim\limits_{n\to\infty} 1\right)\right) \\\overset{H}{=} 2\exp\left(\lim\limits_{n\to\infty} n \cdot \left(\lim\limits_{n\to\infty}\frac{\frac{1}{n}}{\frac{\ln 2}{2\sqrt n}} - \lim\limits_{n\to\infty} 1\right)\right) = 2\exp\left(\lim\limits_{n\to\infty} n \cdot \left(\lim\limits_{n\to\infty}\frac{2\sqrt n}{n\ln 2} - \lim\limits_{n\to\infty} 1\right)\right) \\= 2\exp\left(\lim\limits_{n\to\infty} n \cdot \left(\lim\limits_{n\to\infty}\frac{2}{\sqrt n\ln 2} - \lim\limits_{n\to\infty} 1\right)\right){ = 2\exp\left( \infty \cdot \left( 0 - 1 \right)\right) = 2\exp\left(-\infty)\right)} = 0$\newline
%
%$ f_2 \preccurlyeq f_3 $
%
%\noindent$ \lim\limits_{n\to\infty} \frac{f_2(n)}{f_3(n)} = \lim\limits_{n\to\infty} \frac{{2^n}}{10^n2^{n/2}} = \lim\limits_{n\to\infty} \frac{2^n}{10^n{\sqrt2}^n} = \lim\limits_{n\to\infty} {\left(\frac{2}{10{\sqrt2}}\right)}^n = \lim\limits_{n\to\infty} {\left(\frac{1}{5{\sqrt2}}\right)}^n = 0$\newline

\end{proof}

























\end{document}